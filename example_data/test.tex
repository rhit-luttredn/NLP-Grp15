\documentclass[11pt]{article}
\usepackage{amsmath,amsthm,amssymb,scrextend,mathtools}

\usepackage{enumitem, graphicx}
\usepackage{hhline}

%this is for drawing finite automata state diagrams etc. see example below in question #2.
\usepackage{tikz}
\usetikzlibrary{automata, positioning, arrows,chains,scopes,fit}
\tikzset{
->, % makes the edges directed
>=stealth', % makes the arrow heads bold
node distance=2.4cm, % specifies the minimum distance between two nodes. Change if necessary.
every state/.style={thick, fill=gray!10}, % sets the properties for each ’state’ node
initial text=$ $, % sets the text that appears on the start arrow
}

\textheight9.5in \textwidth6.5in \oddsidemargin0in
\evensidemargin0in \topmargin-0.75in \sloppy
\setlength{\parindent}{0pt} \setlength{\parskip}{8pt}
\pagestyle{empty} \setcounter{section}{1}
\DeclarePairedDelimiter\set\{\}


%%%%%%%%%%%%% end of definition for command \dotminus

\newtheorem{theorem}{Theorem}
\newtheorem{lemma}[theorem]{Lemma}
\newtheorem{cor}[theorem]{Corollary}

\newenvironment{defn}[1][Definition]{\begin{trivlist}
\item[\hskip \labelsep {\bfseries #1}]}{\end{trivlist}}

\newenvironment{ex}[1][Example]{\begin{trivlist}
\item[\hskip \labelsep {\bfseries #1}]}{\end{trivlist}}

%\newcommand {\qed}{\mbox{$\Box$}}
\renewcommand {\iff}{\Longleftrightarrow}
\newcommand {\R}{\mathbb{R}}
\newcommand {\N}{\mathbb{N}}
\newcommand {\Q}{\mathbb{Q}}
\newcommand {\Z}{\mathbb{Z}}
\newcommand {\A}{\mathcal{A}}
\newcommand {\B}{\mathcal{B}}
\newcommand {\C}{\mathcal{C}}
\newcommand {\F}{\mathcal{F}}
\newcommand {\G}{\mathcal{G}}
\newcommand {\U}{\mathcal{U}}
\newcommand {\sub}{\mbox{SB}}
\newcommand {\dist}{\mbox{dist}}
\newcommand {\p}{\mathcal{P}}
\newcommand{\Dom}{\mathrm{Dom}}
\newcommand{\Ran}{\mathrm{Ran}}

\begin{document}

\begin{center}
 
  \LARGE{HW 8}\\
  \Large{MA376-01 \\ }
  \Large{Zach Humes}

\end{center}
% {\bf Please read the following homework guidelines before starting.}

% \begin{itemize}
% \item Get started ON TIME on all assignments! You'll have time to get questions answered if you don't put things off to the last minute.
% \item Assignment submissions are to be typed or written neatly. Using \LaTeX{} is optional but encouraged.
% \item Problems should be numbered as they are on this paper (1, 2, 3, etc.)
% \item Problems must appear in order. If you cannot complete a given problem, write the problem number and label, and indicate that you do not have a solution. Otherwise you may waste the grader's time looking for the problem.
% \item Leave adequate spacing. One page per problem is often appropriate, but anything that clearly indicates problem boundaries is acceptable.
% \item When submitting electronically, take good scans and/or photos and upload as a single PDF file.
% \item You are encouraged to discuss homework problems with classmates, TAs, and instructor, but all work that you
% write up should be {\bf your own and in your own words}. 
% You may make free use of the textbook and class notes, but you must {\bf cite any other
% source you use}.
% \end{itemize}

% \begin{document}

{\bf Problems}

\begin{enumerate}
    \item [(1)] Show that if $|G| = 30$ then either $G$ contains a normal subgroup isomorphic to $\Z/5\Z$ or $G$ has 24 elements of order 5 and a normal subgroup isomorphic to $\Z/3\Z$.

    \textbf{Proof.} Let $G$ be as stated. The prime factorization of $|G|$ is $2 \cdot 3 \cdot 5$. By Sylow I, there is a Sylow group for $p = 2, 3, 5$. By Sylow III, 
    \begin{align*}
        n_2 &= 1, 3 \text{ or } 5\\
        n_3 &= 1 \text{ or } 10\\
        n_5 &= 1 \text{ or } 6
    \end{align*}
    Case 1 - $n_5 = 1$. Let $H_5$ denote the 5-Sylow subgroup. Because $n_5 = 1$, $H_5 \trianglelefteq G$. Since $|G| = 5^1 \cdot 6 \implies |H_5| = 5$. Since $|H_5|$ is prime, it is cyclic. Since $H_5$ is cyclic and of order 5, it is isomorphic to $\Z/5\Z$.
    
    Case 2 - $n_5 = 6$. By Sylow II, each 5-Sylow group contains distinct order-5 elements. So $G$ contains 24 distinct elements of order 5. Note that in this case $n_3$ cannot be 10 - Supposing $n_3 = 10 \implies G$ contains 10 distinct elements of order 3, totalling to 34 distinct elements, which is already greater than the stated cardinality of $G$, a contradiction. So $n_3 = 1$. Let $H_3$ denote the 3-Sylow subgroup of $G$. Because $n_3 = 1$, $H_3 \trianglelefteq G$. By similar argument as in case 1, $|G| = 3^1 \cdot 15 \implies |H_3| = 3 \implies H_3$ is isomorphic to $\Z/3\Z$. $\square$
    
    \item [(2)] Show that if $|G| = p^2q$ with $p, q$ distinct primes, then $G$ has a normal Sylow subgroup.

    \textbf{Proof.} Let $G, p, q$ be as stated. By Sylow III,
    $n_p \equiv 1 \text{ mod } p$ where $n_p \mid q$. Since $p \neq q$, and $p, q$ are primes, $p \nmid q$. So $n_p = 1$. Because $n_p = 1$, the $p$-Sylow subgroup of $G$ is normal to $G$. $\square$

    \item [(3)] Find all the 2-Sylow subgroups of $D_{12}$. 

    $N_2 := \langle f_1, f_2f_4^2 \rangle$, $N_3 := \langle f_4 \rangle$. $|N_2| = 2^3 = 8$, $|N_3| = 3$.

    \item [(4)] Show that if $|G| = 56$ then $G$ has a normal Sylow subgroup.

    \textbf{Proof.} The prime factorization of $|G|$ is $|G| = 2^3 \cdot 7$. By Sylow III, $n_7 = 1 \text{ or } 8$.

    Case 1 - $n_7 = 1$. This directly implies that the $7-Sylow$ group is normal in $G$.

    Case 2 - $n_7 = 8$. Then there are 48 distinct 7-order elements of $G$. Let $H_2$ be a 2-Sylow subgroup of $G$. Then $|H_2| = 8$. Let $C_7$ be the collection of all distinct 7-order elements of $G$. Then $|C_7 \cup H_2| = 56 = |G|$. Since $C_7 \cap H_2 = \emptyset$, $C_7 \cup H_2 = G$. Then $G$ has one 2-Sylow group, $H_2$, meaning that $H_2 \trianglelefteq G$. $\square$

    \item [(5)] Show that if $|G| = 2907$ then $G$ is not simple.

    \textbf{Proof.} $|G| = 3^2 \cdot 17 \cdot 19$. By Sylow III, $n_{17} = 1 \text{ or } 171$, $n_{19} = 1, \textbf{ or } 152$. If $n_{17} = 171$, then there are 2736 17-order elements of $G$. Then $n_{19}$ must be 1, otherwise there would be 2888 19-order elements in $G$, and there would be too many elements in total. Similarly, if $n_{19} = 152$, $n_{17}$ must be 1. In any case, $n_p = 1$ for either $p = 17 \text{ or } 19$. Thus, $G$ always has at least one normal 17-Sylow or 19-Sylow subgroup, so $G$ is not simple. $\square$ 

    \item [(6)] Show that if $G$ is a non-abelian simple group of order $< 100$ then $|G| = 60$. 

    \textbf{Proof.} Suppose $G \cong A_5$. Then $|G| = 60$, $G$ is nonabelian and simple. 
    
    Trivially, groups with order $1$ are not simple.
    
    We can eliminate groups with prime order since they are isomorphic to $\Z/p\Z$, which is abelian: \\ $\{2, 3, 5, 7, 11, 13, 17, 19, 23, 29, 31, 37, 41, 43, 47, 53, 59, 61, 67, 71, 73, 79, 83, 89, 97\}$.

    We can eliminate groups with orders that are prime powers since they are $p$-groups and thus have nontrivial centers, centers are normal, so these groups are not simple:\\ $\{ 4, 8, 9, 16,  25, 27, 32, 49, 64, 81\}$.

    Groups with orders that are the products of two distinct primes (semiprimes) are not simple because they always have a normal Sylow subgroup. So we can eliminate groups with the following orders: \\ $\{ 6, 10, 14, 15, 21, 22, 26, 33, 34, 35, 38, 39, 46, 51, 55, 57, 58, 62, 65, 69, 74, 77, 82, 85, 86, 87, 91, 93, $\\
    $94, 95\}$.

    Groups with orders that are the product of a square prime and a distinct prime are not simple as a result of (2). So we can eliminate groups with the following orders:\\ 
    $\{12, 18, 20, 28, 44, 45, 50, 52, 63, 75, 76, 92, 98, 99\}$.
    
    Groups with order 24 are not simple as a result in the lecture 30 notes. In class, we demonstrated that groups with the following orders are not simple and/or abelian:\\
    $\{36, 40, 48, 56, 72, 80, 88\}$

    I am going to concede the fact that groups with order $90$ and $96$ are either simple/abelian because it was stated as fact in class. 

    By (1), groups of order 30 are not simple.
    
    Since we've exhausted our theorems, previous work, and concessions, we need to investigate the remaining orders:\\ $\{42, 54, 66, 68, 70, 78, 84\}$

    Let $G_n$ denote a group with order $n$. 
    
    Consider $G_{42}$. The prime factorization of $42$ is $7 \cdot 3 \cdot 2$. By Sylow III, $n_{7} = 1$. So $G_{42}$ has a single 7-Sylow subgroup, implying that $G_{42}$ is not simple. The same argument holds for $G_{84}$. Consider $G_{54}$. The prime factorization of 54 is $2 \cdot 3^3$. A similar argument holds for $G_{54}$ - by Sylow III, $n_3 = 1$. For $G_{66}$, $n_{11} = 1$. For $G_{68}$, $n_{17} = 1$. For $G_{70}$, $n_7 = 1$. For $G_{78},$ $n_{13} = 1$. Finally, for $G_{84}$, $n_7 = 1$. So we can conclude that groups with the orders that could not be eliminated directly from theorems cannot be simple.
    
    Because there are no instances of nonabelian simple groups other than $G \cong A_5$ with order $< 100$, the claim holds. $\square$
    
\end{enumerate}

\end{document}
